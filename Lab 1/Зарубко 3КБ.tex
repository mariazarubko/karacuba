\documentclass[11pt]{article}
\usepackage{amsmath,amssymb,amsthm}
\usepackage{algorithm}
\usepackage[noend]{algpseudocode} 

%---enable russian----

\usepackage[utf8]{inputenc}
\usepackage[russian]{babel}

\usepackage{biblatex}
\addbibresource{sample.bib}

% PROBABILITY SYMBOLS
\newcommand*\PROB\Pr 
\DeclareMathOperator*{\EXPECT}{\mathbb{E}}


% Sets, Rngs, ets 
\newcommand{\N}{{{\mathbb N}}}
\newcommand{\Z}{{{\mathbb Z}}}
\newcommand{\R}{{{\mathbb R}}}
\newcommand{\Zp}{\ints_p} % Integers modulo p
\newcommand{\Zq}{\ints_q} % Integers modulo q
\newcommand{\Zn}{\ints_N} % Integers modulo N

% Landau 
\newcommand{\bigO}{\mathcal{O}}
\newcommand*{\OLandau}{\bigO}
\newcommand*{\WLandau}{\Omega}
\newcommand*{\xOLandau}{\widetilde{\OLandau}}
\newcommand*{\xWLandau}{\widetilde{\WLandau}}
\newcommand*{\TLandau}{\Theta}
\newcommand*{\xTLandau}{\widetilde{\TLandau}}
\newcommand{\smallo}{o} %technically, an omicron
\newcommand{\softO}{\widetilde{\bigO}}
\newcommand{\wLandau}{\omega}
\newcommand{\negl}{\mathrm{negl}} 

% Misc
\newcommand{\eps}{\varepsilon}
\newcommand{\inprod}[1]{\left\langle #1 \right\rangle}


\newcommand{\handout}[5]{
	\noindent
	\begin{center}
		\framebox{
			\vbox{
				\hbox to 5.78in { {\bf Научно-исследовательская практика} \hfill #2 }
				\vspace{4mm}
				\hbox to 5.78in { {\Large \hfill #5  \hfill} }
				\vspace{2mm}
				\hbox to 5.78in { {\em #3 \hfill #4} }
			}
		}
	\end{center}
	\vspace*{4mm}
}

\newcommand{\lecture}[4]{\handout{#1}{#2}{#3}{Scribe: #4}{Быстрое умножение методом Карацубы #1}}

\newtheorem{theorem}{Теорема}
\newtheorem{lemma}{Лемма}
\newtheorem{definition}{Определение}
\newtheorem{corollary}{Следствие}
\newtheorem{fact}{Факт}

% 1-inch margins
\topmargin 0pt
\advance \topmargin by -\headheight
\advance \topmargin by -\headsep
\textheight 8.9in
\oddsidemargin 0pt
\evensidemargin \oddsidemargin
\marginparwidth 0.5in
\textwidth 6.5in

\parindent 0in
\parskip 1.5ex

\begin{document}
	
	\lecture{}{Лето 2020}{}{Зарубко Мария Владимировна}
	
	\section{Теория}
	Алгоритм Карацубы- метод быстрого умножения чисел. В отличие от обычного способа умножения, который имеет вычислительную сложность \(O(n^2)\), алгоритм Карацубы требует только $n^{\log_{2}{3}}$ операций. Пусть каждое из рассматриваемых десятичных чисел A и B разбиваются на два числа длины n и тогда их можно представить как:
	
	$A=a\cdot10^n+b$
	
	$B=c\cdot10^n+d$
	
	При наивном умножении $AB=a\cdot c\cdot 10^{2\cdot n}+(a\cdot d+c\cdot b)\cdot 10^n+b\cdot d$, то есть необходимо вычислить $a\cdot c,b\cdot d,c\cdot b,a\cdot d$, поэтому сложность алгоритма будет $4\cdot n^2$. При умножении Карацубы необходимо найти $a\cdot c\cdot x^2+((a+b)\cdot (d+c)-a\cdot c-b\cdot d)\cdot x+b\cdot d$, таким образом вычисляются только $(a+b)\cdot (d+c), b\cdot d, a\cdot c$, что уменьшает сложность алгоритма до $3\cdot n^2$.
	
	\section{Алгоритм}
	
	\begin{algorithm}
		\caption{Алгоритм Карацубы}\label{karatsuba}
		\begin{algorithmic}[1]
			\Procedure{Karatsuba}{$A,B$}
			\State 1. Разложим A и B следующим образом:
			\State  \ \ \ \ \ \ 1.1 $A\gets a\cdot x+b$
			\State  \ \ \ \ \ \ 1.2 $B\gets c\cdot x+d$\Comment{Где \(x=m^{n/2}\), n - четное и m - степень системы счисления}
			\State  \ \ \ \ \ \ 1.3 $AB\gets a\cdot c\cdot x^2+((a+b)\cdot(d+c)-a\cdot c-b\cdot d)x+b\cdot d$
			\\ \ \ \ \ \ \ 2. Если \(a,b,c,d\) могут быть разложены таким же образом как и \(A,B\), то вернуться на шаг 1.1 
			\State \textbf{return} $AB$
			\EndProcedure
		\end{algorithmic}
	\end{algorithm}
	
	Данный алгоритм не дает существенного преимущества при вычислении чисел малых длин, но становится намного эффективнее при вычислении чисел порядка десятков десятичных разрядов.
	
	\section{Сравнение работы алгоритмов}
	
	\subsection{Описание машины, использованной для тестов}
	Характеристики машины следующие:
	
	\textbf{Процессор}: Intel(R) Core(TM) i5-8265U CPU @ 1.60GHz 1.80 GHz
	
	\textbf{Оперативная память}: 8,00 ГБ
	
	\textbf{Видеоадаптер}: Intel(R) UHD Graphics 620
	
	\textbf{Модель материнской платы}: X509FA, ASUSTeK COMPUTER INC.
	
	\subsection{Анализ}
	\begin{tabular}{|c|l|}
		\hline
		№ & Значение x  \\
		\hline
		1 & \(1566156757236357352743265742^{110}\) \\
		\hline
		2 & \(987657565464456645689957352743265742^{222}\) \\
		\hline
		3 & \(987658437987985843657835983443287890098465673765765757657576249837^{1345}\) \\
		\hline
	\end{tabular}
	
	\begin{tabular}{|c|l|}
		\hline
		№ & Значение y \\
		\hline
		1 & \(3487382686478324643475834678^{110}\) \\
		\hline
		2 & \(878765362589097888999489283775834678^{222}\) \\
		\hline
		3 & \(983247325463743748389924632246746378445654646632342368987475255665^{1345}\) \\
		\hline
	\end{tabular}
	
	\begin{tabular}{|c|c|c|}
		\hline
		№ & Скорость работы алгоритма & Скорость работы встроенной функции \\
		\hline
		1 & 2.01 s & 142 $\mu$s \\
		\hline
		2 & 8.30 s & 9.19 ms \\
		\hline
		3 & 256.80 s & 7.65 ms \\
		\hline
	\end{tabular}
	
	\cite{wiki}
	\cite{lvov}
	
	\printbibliography[title={Bibliography}]
	
\end{document}
