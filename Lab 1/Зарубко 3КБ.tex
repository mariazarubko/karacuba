\documentclass[11pt]{article}
\usepackage{amsmath,amssymb,amsthm}
\usepackage{algorithm}
\usepackage[noend]{algpseudocode} 

%---enable russian----

\usepackage[utf8]{inputenc}
\usepackage[russian]{babel}

% PROBABILITY SYMBOLS
\newcommand*\PROB\Pr 
\DeclareMathOperator*{\EXPECT}{\mathbb{E}}


% Sets, Rngs, ets 
\newcommand{\N}{{{\mathbb N}}}
\newcommand{\Z}{{{\mathbb Z}}}
\newcommand{\R}{{{\mathbb R}}}
\newcommand{\Zp}{\ints_p} % Integers modulo p
\newcommand{\Zq}{\ints_q} % Integers modulo q
\newcommand{\Zn}{\ints_N} % Integers modulo N

% Landau 
\newcommand{\bigO}{\mathcal{O}}
\newcommand*{\OLandau}{\bigO}
\newcommand*{\WLandau}{\Omega}
\newcommand*{\xOLandau}{\widetilde{\OLandau}}
\newcommand*{\xWLandau}{\widetilde{\WLandau}}
\newcommand*{\TLandau}{\Theta}
\newcommand*{\xTLandau}{\widetilde{\TLandau}}
\newcommand{\smallo}{o} %technically, an omicron
\newcommand{\softO}{\widetilde{\bigO}}
\newcommand{\wLandau}{\omega}
\newcommand{\negl}{\mathrm{negl}} 

% Misc
\newcommand{\eps}{\varepsilon}
\newcommand{\inprod}[1]{\left\langle #1 \right\rangle}

 
\newcommand{\handout}[5]{
  \noindent
  \begin{center}
  \framebox{
    \vbox{
      \hbox to 5.78in { {\bf Научно-исследовательская практика} \hfill #2 }
      \vspace{4mm}
      \hbox to 5.78in { {\Large \hfill #5  \hfill} }
      \vspace{2mm}
      \hbox to 5.78in { {\em #3 \hfill #4} }
    }
  }
  \end{center}
  \vspace*{4mm}
}

\newcommand{\lecture}[4]{\handout{#1}{#2}{#3}{Scribe: #4}{Быстрое умножение методом Карацубы #1}}

\newtheorem{theorem}{Теорема}
\newtheorem{lemma}{Лемма}
\newtheorem{definition}{Определение}
\newtheorem{corollary}{Следствие}
\newtheorem{fact}{Факт}

% 1-inch margins
\topmargin 0pt
\advance \topmargin by -\headheight
\advance \topmargin by -\headsep
\textheight 8.9in
\oddsidemargin 0pt
\evensidemargin \oddsidemargin
\marginparwidth 0.5in
\textwidth 6.5in

\parindent 0in
\parskip 1.5ex

\begin{document}

\lecture{}{Лето 2020}{}{Зарубко Мария Владимировна}

\section{Теория}
Алгоритм Карацубы- метод быстрого умножения чисел. В отличие от обычного способа умножения, который имеет вычислительную сложность \(O(n^2) \), алгоритм Карацубы требует только \(n^{log_2 3}\) операций. Пусть каждое из рассматриваемых десятичных чисел A и B разбиваются на два числа длины n и тогда их можно представить как:

\(A=a*10^n+b\)

\(B=c*10^n+d\)

При наивном умножении \(A*B=a*c*10^{2*n}+(a*d+c*b)*10^n+b*d\), то есть необходимо вычислить \(a*c,b*d,c*b,a*d\), поэтому сложность алгоритма будет \(4*n^2\). При умножении Карацубы необходимо найти \(a*c*x^2+((a+b)(d+c)-a*c-b*d)x+b*d\), таким образом вычисляются только \((a+b)(d+c), b*d, a*c\), что уменьшает сложность алгоритма до \(3*n^2\).

\section{Алгоритм}

\begin{algorithm}
	\caption{Алгоритм Карацубы}\label{karatsuba}
	\begin{algorithmic}[1]
		\Procedure{Karatsuba}{$A,B$}
		\State 1. Разложим A и B следующим образом:
		\State  \ \ \ \ \ \ 1.1 $A\gets ax+b$
		\State  \ \ \ \ \ \ 1.2 $B\gets cx+d$\Comment{Где \(x=m^{n/2}\), n - четное и m - степень системы счисления}
		\State  \ \ \ \ \ \ 1.3 $AB\gets acx^2+((a+b)(d+c)-ac-bd)x+bd$
		\\ \ \ \ \ \ \ 2. Если \(a,b,c,d\) могут быть разложены таким же образом как и \(A,B\), то вернуться на шаг 1.1 
		\State \textbf{return} $AB$
		\EndProcedure
	\end{algorithmic}
\end{algorithm}

Данный алгоритм не дает существенного преимущества при вычислении чисел малых длин, но становится намного эффективнее при вычислении чисел порядка десятков десятичных разрядов.

\section{Сравнение работы алгоритмов}

\subsection{Описание машины, использованной для тестов}
Характеристики машины следующие:

\textbf{Процессор}: Intel(R) Core(TM) i5-8265U CPU @ 1.60GHz 1.80 GHz

\textbf{Оперативная память}: 8,00 ГБ

\textbf{Видеоадаптер}: Intel(R) UHD Graphics 620

\textbf{Модель материнской платы}: X509FA, ASUSTeK COMPUTER INC.

\subsection{Анализ}
\begin{tabular}{|c|l|}
	\hline
	№ & Значение x  \\
	\hline
	1 & \(1566156757236357352743265742^{110}\) \\
	\hline
	2 & \(987657565464456645689957352743265742^{222}\) \\
	\hline
	3 & \(987658437987985843657835983443287890098465673765765757657576249837^{1345}\) \\
	\hline
\end{tabular}

\begin{tabular}{|c|l|}
	\hline
	№ & Значение y \\
	\hline
	1 & \(3487382686478324643475834678^{110}\) \\
	\hline
	2 & \(878765362589097888999489283775834678^{222}\) \\
	\hline
	3 & \(983247325463743748389924632246746378445654646632342368987475255665^{1345}\) \\
	\hline
\end{tabular}

\begin{tabular}{|c|c|c|}
	\hline
	№ & Скорость работы алгоритма & Скорость работы встроенной функции \\
	\hline
	1 & 556 µs & 33.4 µs \\
	\hline
	2 & 2.13 ms & 108 µs \\
	\hline
	3 & 101 ms & 3.26 ms \\
	\hline
\end{tabular}

\paragraph{Bibliography.}
\leavevmode\\ $[1]$ Wikipedia: Wikipedia, The Free Encyclopedia [online], http://www.wikipedia.org, June 2020
\leavevmode\\ $[2]$ Львовский С.М. - Набор и верстка в системе LATEX. 5-е изд. 2014.

\end{document}